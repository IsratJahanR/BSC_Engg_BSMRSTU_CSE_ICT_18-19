\documentclass[a4paper, 12pt]{article}

\usepackage[margin=1in]{geometry}
\usepackage{blindtext}
\usepackage{array}
\newcolumntype{P}[1]{>{\centering\arraybackslash}p{#1}}
\newcolumntype{M}[1]{>{\centering\arraybackslash}m{#1}}

\begin{document}

\title{\Large{\textbf{Artificial Intelligence}}}
\author{Mostofa Auntu}
\date{\today}
\maketitle

\tableofcontents

\section{What Is Artificial Intelligence}
Artificial intelligence (AI) is wide-ranging branch of computer science concerned with building smart machines capable of performing tasks the typically require human intelligence. AI is an interdisciplinary science with multiple approaches, but advancements in machine learning and deep learning are creating a paradigm shift in virtually every sector of the tech industry.\\ \\
As machines become increasingly capable, tasks considered to require "intelligence" are often removed from the definition of AI, a phenomenon known as the AI effect. A quip in Tesler's Theorem says "AI is whatever hasn't been done yet." For instance, optical character recognition is frequently excluded from things considered to be AI, having become a routine technology. Modern machine capabilities generally classified as AI include successfully understanding human speech, competing at the highest level in strategic game systems (such as chess and Go), autonomously operating cars, intelligent routing in content delivery networks, and military simulations.

$$F = T \nabla S_T$$
\begin{center} The formula for intelligence \end{center}

\section{Difference between AI and Machine Learning}
In computer science, artificial intelligence (AI) is sometimes called machine intelligence. But it is not the same as "Machine learning". Here are some differences between them:-

\begin{table}
	\centering
	\begin{tabular}{| M{3in} | M{3in} |}
	\hline
	\textbf{Artificial Intelligence} & \textbf{Machine Learning} \\ \hline
	Artificial intelligence originated around 1950s. & Machine learning originated around 1960s \\ \hline
	AI represents simulated intelligence in machines & Machine Learning is the practice of getting machines to make decisions without being programmed \\ \hline
	AI is a subset of Data Science. & Machine learning is a subset of AI \& Data Science \\ \hline
	Aim is build machines which are capable of thinking like humans. & Aim is to make machines learn through data so that they can solve problems. \\ \hline
	\end{tabular}
\end{table}

\section{Why Is Artificial Intelligence Important}
\begin{itemize}
\renewcommand{\labelitemi}{$\star$}
\item AI automates repetitive learning and discovery through data. But AI is different from hardware-driven, robotic automation. Instead of automating manual tasks, AI performs frequent, high-volume, computerized tasks reliably and without fatigue. For this type of automation, human inquiry is still essential to set up the system and ask the right questions.

\item AI adds intelligence to existing products. In most cases, AI will not be sold as an individual application. Rather, products you already use will be improved with AI capabilities, much like Siri was added as a feature to a new generation of Apple products. Automation, conversational platforms, bots and smart machines can be combined with large amounts of data to improve many technologies at home and in the workplace, from security intelligence to investment analysis.

\item AI adapts through progressive learning algorithms to let the data do the programming. AI finds structure and regularities in data so that the algorithm acquires a skill: The algorithm becomes a classifier or a predictor. So, just as the algorithm can teach itself how to play chess, it can teach itself what product to recommend next online. And the models adapt when given new data. Back propagation is an AI technique that allows the model to adjust, through training and added data, when the first answer is not quite right.

\item AI analyzes more and deeper data using neural networks that have many hidden layers. Building a fraud detection system with five hidden layers was almost impossible a few years ago. All that has changed with incredible computer power and big data. You need lots of data to train deep learning models because they learn directly from the data. The more data you can feed them, the more accurate they become.

\item AI achieves incredible accuracy through deep neural networks which was previously impossible. For example, your interactions with Alexa, Google Search and Google Photos are all based on deep learning and they keep getting more accurate the more we use them. In the medical field, AI techniques from deep learning, image classification and object recognition can now be used to find cancer on MRIs with the same accuracy as highly trained radiologists.

\item AI gets the most out of data. When algorithms are self-learning, the data itself can become intellectual property. The answers are in the data; you just have to apply AI to get them out. Since the role of the data is now more important than ever before, it can create a competitive advantage. If you have the best data in a competitive industry, even if everyone is applying similar techniques, the best data will win.
\end{itemize}

\section{Different Types Of AI}
\begin{enumerate}
\item \textbf{Reactive Machines AI:} Based on present actions, it cannot use previous experiences to form current decisions and simultaneously update their memory. Example: Deep Blue
\item \textbf{Limited Memory AI:} Used in self-driving cars. They detect the movement of vehicles around them constantly and add it to their memory.
\item \textbf{Theory of Mind AI:} Advanced AI that has the ability to understand emotions, people and other things in the real world.
\item \textbf{Self Aware AI:} AIs that posses human-like consciousness and reactions. Such machines have the ability to form self-driven actions.
Artificial Narrow Intelligence (ANI): General purpose AI, used in building virtual assistants like Siri.
\item \textbf{Artificial General Intelligence (AGI):} Also known as strong AI. An example is the Pillo robot that answers questions related to health.
\item \textbf{Artificial Superhuman Intelligence (ASI):} AI that possesses the ability to do everything that a human can do and more. An example is the Alpha 2 which is the first humanoid ASI robot.
\end{enumerate}

\section{How Artificial Intelligence Is Being Used}
Every industry has a high demand for AI capabilities – especially question answering systems that can be used for legal assistance, patent searches, risk notification and medical research. Other uses of AI include:

\begin{enumerate}
\item \textbf{Health Care:} AI applications can provide personalized medicine and X-ray readings. Personal health care assistants can act as life coaches, reminding you to take your pills, exercise or eat healthier.
\item \textbf{Retail:} AI provides virtual shopping capabilities that offer personalized recommendations and discuss purchase options with the consumer. Stock management and site layout technologies will also be improved with AI.
\item \textbf{Manufacturing:} AI can analyze factory IoT data as it streams from connected equipment to forecast expected load and demand using recurrent networks, a specific type of deep learning network used with sequence data.
\item \textbf{Banking:} Artificial Intelligence enhances the speed, precision and effectiveness of human efforts. In financial institutions, AI techniques can be used to identify which transactions are likely to be fraudulent, adopt fast and accurate credit scoring, as well as automate manually intense data management tasks.
\end{enumerate}

\end{document}