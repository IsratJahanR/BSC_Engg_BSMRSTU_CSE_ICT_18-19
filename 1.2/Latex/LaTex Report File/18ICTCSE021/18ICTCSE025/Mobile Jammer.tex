\documentclass{article}
\usepackage[margin=1in]{geometry}
\usepackage{graphicx}
\author{Sabbir Hossain Shanto}
\title{Everything about Mobile Jammer}
\date{}
\begin{document}
\pagenumbering{gobble}
\maketitle
\newpage
\pagenumbering{arabic}
\section{Introduction}
A \textbf{mobile phone jammer} is a device which deliberately transmits signals on the same radio frequencies as mobile phones, disrupting the communication between the phone and the cell-phone base station, effectively disabling mobile phones within the range of the jammer, preventing them from receiving signals and from transmitting them. Jammers can be used in practically any location, but are found primarily in places where a phone call would be particularly disruptive because silence is expected, such as entertainment venues.
\bigskip
\includegraphics[width=\linewidth]{{"mobile jammer"}.jpg}
\label{Mobile Jammer}
\bigskip
Because they disrupt the operations of legitimate mobile phone services, the use of such blocking devices is \emph{illegal} in many jurisdictions, especially without a licence. When operational, such devices also block access to emergency services.
\section{Materials and Methods}
\subsection{Materials}
Jammer-circuit For any jammer-circuit, there  are  three  main-important-functional parts,  and  when  they  are  combined  together, the output will work as a jammer. 
There are: 
\begin{enumerate}
\item RF-amplifier
\item Voltage-controlled-oscillator
\item Tuning-circuit
\end{enumerate}
\subsection{Methods}
The components used for the jammer, including their- relevant- details are shown in Table 1.  In  addition,  transmitting-antenna  is  necessary,  as  most-important-part  of any-transmitter.  In  order  to have  optima-power-transfer,  the  antenna-system  must  be  matched  to  the  transmission-system.  The  main-characteristic  of  antenna  is  VSWR  (Voltage-Standing-Wave-Ratio).  The antenna  should  have  VSWR  of  $3$ or lower, because the return-loss of this antenna is minimal. The antenna used in the project is $4$ wave monopole-antenna  and  it has  $50$ Ohm  impedance  so that  the  antenna  is  matched  to  the  transmission  system.  Also this antenna has low  VSWR  of $1.7$, and a bandwidth of 150MHz around  $916$  MHz center- frequencies  which cover the mobile-jammer-frequency-range. The antenna gain is $2$dBi.
\bigskip
\begin{figure}


\includegraphics[width=\linewidth]{{"circuit of a jammer"}.png}
\end{figure}
\bigskip
Frequency Generating Before linking all to the antenna, power-supply shall not be switched-on at first; also the antenna should not be taking-off when the mainframe is in the working-condition. The jammer shall be installed in the position with  good-ventilation,  and  large-scale-things  shall  be avoided  to  ensure  to the-shielding-effect.  When  use  the jammer outdoors, preventing water shall be taken into consideration. The GSM  mobile-phones-used for testing of the jammer are: Samsung- GT-E1080T; Blackberry-7290; Motorolla, C118; Nokia-1100, 1661, 6300; Tecno-T570, T780; TV22i; and iPhonei9+. These-phones were fully-charged so as to avoid the risk of switching-off during the testing-process. 

\section{Result and Discussion}
\subsection{Result}
The simulated-circuit of mobile-phone-jammer is shown in  Figure  4,  while Figure 5 illustrates the Breadboard- Assembly of the jammer.  The jammer is powered by the 9V-battery D.C. The transistor Q1,  capacitors C4 and C5 and resistor R1 constitute  the  RF-amplifier-circuit.  This  will  amplify  the  signal  generated  by  the  tuned-circuit.  The amplification-signal  is  given to  the  antenna  through  C6  capacitor,  which removes  the  DC and  allows only  the AC  signal  to  be transmitted.  When the  transistor  Q1  is  turned  ON,  the tuned-circuit  at the  collector  will  get turned ON. The tuned-circuit consists of capacitor C1 and inductor L1. This tuned-circuit will act as an oscillator with  zero  resistance.  This-oscillator  or  tuned-circuit  will  produce  the  very-high-frequency  with  minimum-damping. The both inductor and capacitor of tuned-circuit will oscillate at its resonating frequency.
The  tuned-circuit  operation  is  as  follows:  When  the  circuit  gets  ON,  the  voltage  is  stored  by  the capacitor according to its capacity. The main-function of capacitor is to store electric energy. Once the capacitor is completely charged, it will allow the charge to flow through inductor, which is used to store magnetic-energy. When  the  current  is  flowing  across  the inductor,  it will  store  the  magnetic-energy  by  this  voltage  across  the capacitor and will get decreased, at some point complete-magnetic-energy is stored by inductor and the charge or voltage  across  the  capacitor  will  be  zero.  The  magnetic-charge  through  the  inductor  will  decreased  and  the current  will  charge  the  capacitor  in  opposite  or  reverse-polarity-manner.  Again  after  some-period  of  time, capacitor will get  completely- charged  and  magnetic-energy across the  inductor  will be completely zero. Again the capacitor  will  give charge  to  the  inductor and becomes zero.  After  some  time,  inductor will give charge to capacitor and become zero and they will oscillate and generate the frequency. This circle run up  to the internal-resistance is  generated  and oscillations will get stop.  RF-amplifier  feed is given through the  capacitor C5 to the collector terminal before C6 for gain or like a boost-signal to the tuned-circuit-signal. The capacitors C2 and C3 are  used  for  generating the  noise for  the  frequency  generated by  the  tuned-circuit.  Capacitors C2  and C3  will generate the electronic-pulses in some random fashion, so-called  modulating signals (noise). The feedback-back or boost  given  by  the  RF-amplifier-frequency  generated  by  the tuned-circuit, the  noise-signal  generated by the capacitors C2 and C3 will be combined, amplified and transmitted to the air.
\subsection{Legal Issues}
The  review-recorded-below do  not  claim to  be  fully-comprehensive-account  of  every-instance  associated  with the  Legal-Issues related to RF-signals-jamming, however, the assessment does give a fairly-good-picture of the order  of  magnitude  of  activities,  achievements,  and  problems  encountered,  and  probably  include  the  most significant ones identified for which information was available at the time this study was carried out.  Cell-phone-jammers are illegal in most-countries,  except  to  the  military,  law-enforcement and  certain-governmental- agencies (ACA, 2003), as it is considered  a “Property Theft” because a private-service-provider-company  has  purchased the  rights  to  the  radio-spectrum,  and  jamming the  spectrum  is  a  kind of  stealing  the property the company has  purchased.  It also represents a “Safety Hazard” because jamming blocks all the  calls in the area, not just the illegal or annoying ones. Jamming a signal could block the emergency-calls, where there is a  life  and  death  situation.  In  addition,  there  could  be  some  innovative  misuse  or  even  abuse  of  the  mobile-phone-jamming-technology,  for  example,  there  has  been  an  extensive-recent-chitchat  on  Twitter,  that  5-star-hotel-chains installed mobile-phone-jammers  to  block  guests’ cell-phone-usage and force  them to  use  in-room-phones at much-higher-rates (personal experience). According  to  the  ACA’s  Declaration  Prohibiting  mobile-phone-jammers  (2003),  currently,  the  most-serious and severe-legal-elimination of mobile-phone-jammers is in Australia, where, for example Section 189 of the Act makes it an offence to operate or supply, or possess for the purposes of operation or supply, a prohibited device, without reasonable excuse. Section 189 also details the penalties that apply if a person is found guilty: if the  offender  is  an  individual  –  imprisonment  for  two  years;  or  otherwise  –  1,500  penalty  units  (currently 165,000 dollar). The  reasons  for  the  prohibition  included:  mobile-phone-jammers  cause  deliberate-interference  to licensed-services and may cause interference to other-services operating in adjacent-spectrum-bands; All mobile-phones  being  used  within  a  radius  of  up  to  four  kilometers  from  the  jamming-device  could  be  ‘jammed’; Concern  that  radiation-levels  of high-powered-devices  may  result  in  human-exposure  to  levels of  EMR,  that exceed  the  maximum permitted  under Australian-health-exposure-standards.  This  has  implications for  public-health and  safety,  especially  in  confined  areas;  Jamming  would  be  likely,  among  other  things,  to  substantially interfere  with, or  disrupt  or  disturb  public-mobile-phone-services  and  have  serious-adverse-consequences  for public-mobile-phone-users  by  jeopardizing  the  quality  and  extent  of  carrier-services,  preventing  access  to emergency-services and causing inconvenience to or loss of business for mobile-phone-users. Other  services  likely  to  be  affected  by  jammers:  Examples  of  the  types  of  radio-communications services operating in bands near those used by mobile-phones and potentially-affected by mobile-phone-jammers are:  trunked-land-mobile-systems;  fixed point-to-point  links which  carry  anything  from  data  to  multi-channel voice  communications;  sound-outside-broadcast  and  studio-to-transmitter  links;  cordless-telephones; interference  with  electromagnetic-  wave  sensitive-devices  such as  life-support-equipment  in  hospitals  (such  as the apnea-monitor) and those  in airplanes,   and  the  large-number  of devices authorized to operate  under  ACA- class-licenses  (such  as  garage-door-openers  and  wireless  LANs),  emergency  organizations  (such  as  poison-information-centres and other-medical-services) or to the normal-phone-numbers for police, fire and ambulance. Mobile-phones are  increasingly  being  used to request emergency-assistance  from the police, fire  or  ambulance services  in  life-threatening or  time-critical  situations,  for example  during  2002-03,  (or  1,128,339) of  the 3,953,564 genuine-calls to emergency-call-service originated from mobile-phones. There is some-evidence of a potential  for  mobile-phone-jammers  to cause  mobile  phones  to  “lock  up”  and  to  remain-so  after  leaving  the jammed-area until the phone is “reset”  (e.g.  by turning-it-off  and  on-again).  The user  may  be  unaware  that this has occurred and of the need to reset the phone (ACA, 2003). Other-countries are dealing with the issue of whether mobile-phone-jamming should be allowed. There have  been  a number of  positions  taken  by  these  countries;  such  as  in United  Kingdom  (UK),  Ireland,  United States  of  America  (USA),  and  Europe  it  is legally-forbidden.  Canada:  With respect  to  the  use of  jamming- devices  in  connection  with  federal-security  and  law-enforcement-functions  for  national-security-purposes,  an alternative-authorization-process  is,  currently,  under review.  In  Jamaica mobile-phone-jammers  are  used  (with specified-restrictions)  in  prisons.  There  is,  however,  a  media-report  which  suggests  that  legitimate-services outside the prison  boundaries are  affected.  It has  also  been reported  that  universities in  Italy  have  adopted  the jamming-technology to prevent overwhelming-cheating, as  the students were openly-taking photos of tests with their camera-phones and sending them to classmates (ACA, 2003).  From  the  above  it  can  be  ironically-perceived,  that  cell-phone-jamming-technology  is,  simply,  an illegal-technology,  which  causes  more-problems  than  it  solves.  In  the  local-context,  however,  Safaricom Company  (the  largest-mobile-service-provider  in  Kenya)  of  Vodafone  group  and  Kenya-Prisons-Services recently announced (after several-pilot-studies) that they will install phone-jamming-equipment in all the major-prisons  (CCK,  2014).  This  was  termed  as  a  response  to  the  runaway-crime  involving  mobile-phones  that  is perpetuated by prisoners.  The strategy of jamming-mobile-phone-signals in prison compounds is a logical-technical-response. By creating islands of non-connectivity in these-jails, it is possible to mitigate the economic and social-risk posed by these  incarcerated-criminals.  CellAntenna  states  that  jammers  provide  the-best  and  most-economical-way  to prevent  cell-  phone use  in prisons,  require  very-little-staff-time, and  that  the cost  of the  system  depends  on a number of factors such as the size and shape of prison, the area to be covered, and incoming tower signal levels (One  News,  2007).  Cell-phones,  especially  smart-phones,  enable  prison  inmates’  access  internet  and  social-media-sites as well as receiving and sending short-messages, and videos which poses challenges to public-safety and rehabilitation (Norris, 2016).  According to survey on inappropriate use and possession of mobile-phones in prisons of Kenya by Ochola  (2015), 34 percent of inmates  reported  to  have  owned  mobile-phones  at  one-given-time, 100 percent of the respondents agreed to have used mobile-phones and 78percent have paid to acquire mobile-phone-usage from  those  inmates  owning  mobile-phones. On  mobile  phone  usage  different-reasons  emerged:  Criminal  acts (swindling the public,  threatening  potential-witnesses and  extortion,  Maintaining contacts with family,  Private-communication  with  minimal-oversight  by  authorities,  Facilitation  of  escapes  and  Arrangement  and  co-ordination  of  contraband  supply  among  others).  Statistics  from  Safaricom  indicate  that  most  of  the  phone-related-fraud-cases originate from prisons; with Kamiti- prison taking the lead with about 1,500 fraudulent SMS and calls  during  one  month only,  which  translates to  65  percent of  the  total-incidents  during  the  month.  Other than Kamiti the practice is also ripe in other-prisons across the country including, Nakuru, Meru, Kibos and Shimola Tewa.  Some-jailbirds  arrange  for  their  friends  to  throw  mobile-  phones  across  the  wall  of  the  prison  after packing them in  plastic-bags, which is considered as contraband. In another  instance,  a  prisoner  staged  a  ‘nude protest’ after the jail-authorities examined him following suspicion that he  was hiding a mobile SIM card  in the private-areas  of  his-body.  Also,  the  jammers  at  the  Kannur  central  prison  (one  of  the  pilot-projects)  were recently  switched  off  after  the  nearby  residents  seriously-complained  that  it  was  affecting  their  mobile-communication. No need to say that this delighted the Kannur prisoners.
\subsection{Recommendation}
The  aim  of  the  project which  was  to build  a simple-mobile-phone-jammer  is  achieved.  Jamming-technique  is potentially very-useful to disable cell-phone in a  particular-range, but it  should-not  affect  the other base station transmission-systems.  Mobile-jammer can be used in  any-location (subject to  particular  legal-restrictions), but, practically,  in  places  where  a  mobile-phone-use  would-be,  on  the  whole,  harmful,  disruptive,  and  even dangerous,  like  in  prisons.  Overall-recommendation  is  that,  further  and  more  deeper-research  is  needed  to produce  more-sophisticated  and  better-jamming-devices,  as  to  not  affect  the  other  base  station  transmission systems.




\end{document}