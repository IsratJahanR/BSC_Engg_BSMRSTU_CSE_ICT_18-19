\documentclass[11pt]{article}
\usepackage{graphicx}
\usepackage[margin=1.5in,left=1.4in, includefoot]{geometry}

\begin{document}
\begin{titlepage}

\textbf{Bangabandhu Sheikh Mujibur Rahman Science and Technology}
\begin{center}
\textbf{University\\}

\bigskip
\bigskip
\bigskip
\medskip
\bigskip
\bigskip
\medskip

Technical writing and presentation\\
\line(1,0){150}\\
[5mm]
\huge {\bfseries Report\\}

\line(1,0){100}\\
[.5mm]
\textsc{\Large Database Management System}

\end{center} 
\vfill

\begin{flushright}

\textbf{\large Shakil Ahamed\\}
Dept. of CSE\\ 
ID: 18ICTCSE064\\

\end{flushright}
\end{titlepage}

\section*{Database Management System\\}
\textbf{Database}is a collection of related data and data is a collection of facts and figures that can be processed to produce information.\\
Mostly data represents recordable facts. Data aids in producing information, which is based on facts. For example, if we have data about marks obtained by all students, we can then conclude about toppers and average marks.\\
A \textbf{database management system} stores data in such a way that it becomes easier to retrieve, manipulate, and produce information.\\

\section*{Characteristics}

Traditionally, data was organized in file formats. DBMS was a new concept then, and all the research was done to make it overcome the deficiencies in traditional style of data management. A modern DBMS has the following characteristics −\\
\subsection*{Real-world entity-} A modern DBMS is more realistic and uses real-world entities to design its architecture. It uses the behavior and attributes too. For example, a school database may use students as an entity and their age as an attribute.

\subsection*{Relation-based tables-} DMBS allows entities and relations among them to form tables.A user can understand the architecture of a database just by looking at the table names.\\
\subsection*{Isolation of data and application-} A database system is entirely different than its data. A database is an active entity, whereas data is said to be passive, on which the database works and organizes. DBMS also stores metadata, which is data about data, to ease its own process.\\
\subsection*{Less redundancy-}DBMS follows the rules of normalization, which splits a relation when any of its attributes is having redundancy in values. Normalization is a mathematically rich and scientific process that reduces data redundancy.\\

\subsection*{Consistency-}Consistency is a state where every relation in a database remains consistent. There exist methods and techniques, which can detect attempt of leaving database in inconsistent state. A DBMS can provide greater consistency as compared to earlier forms of data storing applications like file-processing systems.\\

\subsection*{Query Language-}DBMS is equipped with query language, which makes it more efficient to retrieve and manipulate data. A user can apply as many and as different filtering options as required to retrieve a set of data. Traditionally it was not possible where file-processing system was used.\\

\subsection*{ACID Properties-}DBMS follows the concepts of Atomicity, Consistency, Isolation, and Durability (normally shortened as ACID). These concepts are applied on transactions, which manipulate data in a database. ACID properties help the database stay healthy in multi-transactional environments and in case of failure.\\
\subsection*{Multiuser and Concurrent Access-}DBMS supports multi-user environment and allows them to access and manipulate data in parallel. Though there are restrictions on transactions when users attempt to handle the same data item, but users are always unaware of them.\\

\subsection*{Multiple views-}DBMS offers multiple views for different users. A user who is in the Sales department will have a different view of database than a person working in the Production department. This feature enables the users to have a concentrate view of the database according to their requirements.\\

\subsection*{Security -}Features like multiple views offer security to some extent where users are unable to access data of other users and departments. DBMS offers methods to impose constraints while entering data into the database and retrieving the same at a later stage. DBMS offers many different levels of security features, which enables multiple users to have different views with different features. For example, a user in the Sales department cannot see the data that belongs to the Purchase department. Additionally, it can also be managed how much data of the Sales department should be displayed to the user. Since a DBMS is not saved on the disk as traditional file systems, it is very hard for miscreants to break the code.\\

\section*{Users}

A typical DBMS has users with different rights and permissions who use it for different purposes. Some users retrieve data and some back it up. The users of a DBMS can be broadly categorized as follows −

\subsection*{1. Administrators:}
Administrators maintain the DBMS and are responsible for administrating the database. They are responsible to look after its usage and by whom it should be used. They create access profiles for users and apply limitations to maintain isolation and force security. Administrators also look after DBMS resources like system license, required tools, and other software and hardware related maintenance.\\
\subsection*{2. Designers:}
Designers are the group of people who actually work on the designing part of the database. They keep a close watch on what data should be kept and in what format. They identify and design the whole set of entities, relations, constraints, and views.\\
\subsection*{3. End Users:}
End users are those who actually reap the benefits of having a DBMS. End users can range from simple viewers who pay attention to the logs or market rates to sophisticated users such as business analysts.\\



\end{document}